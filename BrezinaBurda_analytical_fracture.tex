\documentclass[a4paper,10pt]{article}

\usepackage{amsmath}
\usepackage{amsfonts}
\usepackage{amssymb}
\usepackage{stmaryrd}
\usepackage{color}

\newtheorem{prop}{Proposition}

\def\Laplace{\Delta}
\def\prtl{\partial} % partial deriv.
\def\grad{\nabla}
\def\vc#1{\mathbf{\boldsymbol{#1}}}     % vector
\def\todo#1{{TODO:\color{red}#1}}
\def\ol#1{\overline{#1}}

%opening
\title{Analytical solution to a single fracture test problem
%\thanks{This work was supported by the Technology Agency of the Czech Republic under the project no. TA01021331. }
}
\author{%
Jan B\v rezina
\thanks{Technical University in Liberec, Studentsk\'a 2, Liberec, Czech Republic ({\tt jan.brezina@tul.cz}).},
Pavel Burda
\thanks{Czech Technical University, (pavel.burda@fs.cvut.cz)}
}






\begin{document}

\maketitle

\begin{abstract}
We consider a model of the flow in fractured porous media based on the dual continuum approach and 
explicit description of the fracture zones by lower-dimensional objects. For this model, we 
present the mixed-hybrid formulation and discretization using Raviart-Thomas finite elements. 
In particular, we present two new methods for the discrete coupling between equations on meshes of different dimensions
with arbitrary overlapping. Convergence properties of these methods are demonstrated by numerical tests.
\end{abstract}

%\begin{keywords} 
%fracture flow, mortar methods, mixed-hybrid finite elements
%\end{keywords}

%\begin{AMS}
%15A15, 15A09, 15A23
%\end{AMS}

%\pagestyle{myheadings}
%\thispagestyle{plain}
%\markboth{Jan B\v rezina}{Mortar methods and MHFEM for elliptic problems on complex geometries.}


\section{Introduction}
Fractures, cracks, fissures, faults and other discontinuities are ubiquitous in real rock formations especially granitoids.  
Fractures present a challenge for subsurface water flow modeling as they have small volume but large permeability with
potentially large impact on the pressure and velocity fields. Alternatively the filled fractures can behave as barriers. 

Fractures occurs on wide range of scales. The small scale fractures could be homogenized, while the scales comparable to 
the dimensions of a computational domain should be captured explicitly.



small cross-section area. Finally, we obtain system of equations on the domains of different dimension, coupled through the boundary 
conditions. This approach has been proposed by several authors, see e.g. \cite{reichenberger_mixed-dimensional_2006} or
homogenization arguments in \cite{martin_modeling_2005}.  

Even after this simplification, it could be difficult to obtain the mesh where the elements on the fractures match the sides of 
the elements of a surrounding continuum. The aim of this paper is to present two mortar-like methods for mixed-hybrid formulation
that relax this compatibility condition. Since these methods can not capture the jump in the pressure over the fractures, we have
to assume continuous pressure in our model. This is not completely unrealistic, since such a model can be viewed 
as a dual continuum model
in spirit of Gerke and Genuchten \cite{GerkeGenuchten1993a} where the fracture zone is explicitly localized. 




\section{Test problems}
Main result of the paper is derivation of the strong analytical solution for two test problems with coupling between continuum and a fracture.
A Darcy flow is considered  on a square 2D domain $\Omega_2 = (-1,1)\times(-1,1)$
with a horizontal fracture $\Omega_1 = { (x,0) : x\in (-1,1)}$. The fracture splits $\Omega_2$ into upper and 
lower part $\Omega_2^+$ and $\Omega_2^-$, respectively.
The stationary Darcy flow is driven by the same equation on all three domains:
\begin{align}
  \label{eq:Darcy_common}
  -k_2 \Laplace p^+_2(x,y) &= 0 \qquad &&\text{ on }\Omega_2^+,\\
  -k_2 \Laplace p^-_2(x,y) &= 0 \qquad &&\text{ on }\Omega_2^-,\\
  -k_1 p_1''(x) &= f(x) \qquad &&\text{ on }\Omega_1.
  \label{eq:Darcy_common_c}
\end{align}
Where $p_d$, $d=1,2$ is the pressure and $f$ is the communication term that will be specified later. 
We consider constant conductivities $k_2$, $k_1$ on $\Omega_2$ and $\Omega_1$, respectively.

The homogeneous Neumann condition is set on the left and the right side of $\Omega_2$,
while the Dirichlet condition is set on the top and bottom and at the tips of the fracture.
We denote:
\begin{align*}
   \Gamma^N_2 &= \{(x,y)\, :\, x\in\{-1, 1\},\ y\in (-1, 1),\ y\ne 0\},\\
   \Gamma^N_2 &= \{(x,y)\, :\, x\in (-1, 1),\ y\in \{-1, 1\}\},\\
   \Gamma_1   &= \prtl \Omega_1 = \{(-1, 0),\ (1, 0)\}. 
\end{align*}
Then we prescribe following boundary conditions
\begin{align}
  \label{eq:bc_common}
  \prtl_x p_2(x,y) &= 0  &&\text{ on } \Gamma_2^N\\
  p_2(x,y) &= P_2        &&\text{ on } \Gamma_2^D\\
  p_1(x) &= P_1          &&\text{ on } \Gamma_1.
  \label{eq:bc_1}
\end{align}

In order to complete the problem we must prescribe boundary conditions on the fracture and specify the source term $f$. Here we 
distinguish two cases: {\emph a conductive barrier} and {\emph a barrier fracture}.


{\bf Conductive fracture.} In this case we assume a fracture with similar or higher conductivity then in the continuum, $k_1 \ge k_2$.
In such case we can assume continuity of the pressure across the fracture. However we keep difference between $p_2$ and $p_1$ on the fracture.
We set:
\begin{align}
  \label{eq:c_coupling_a}
  p_2^+ &= p_2^-                &&\text{ on }\Omega_1,\\
  -k_2 (-\prtl_y p_2^+ + \prtl_y p_2^-) &= f(x)         &&\text{ on }\Omega_1,\\
  f(x) &= 2\sigma (p_2(x,0) - p_1(x)),   &&
  \label{eq:c_coupling_c}
\end{align}
where $\sigma$ is a coupling parameter, usually $\sigma \approx k_1 / \delta$ with $\delta$ standing for the fracture cross-section. 
Solution of this case is discussed in Section \ref{sec:continuous_frac}.



{\bf Barrier fracture.} Other case is a fracture with significantly smaller conductivity compared to the surrounding continuum. In this case
the pressure $p_2$ is discontinuous across the fracture and two sides of the fracture acts independently. We also 
distinguish conductivities $k_2^+$, $k_2^-$ and boundary pressure $P_2^+$, $P_2^-$ for the upper and lower domains $\Omega_2^+$, $\Omega_2^-$, 
respectively. The coupling on the fracture is prescribed by the boundary conditions:
\begin{align}
  \label{eq:bc_barrier_p}  
  - k_2^+ \grad p_2^+\cdot \vc n^+(x,0) &= k_2^+ \prtl_y p_2^+(x) = f^+(x) 
                    &&\text{ on }\Gamma^{+},\\
  \label{eq:bc_barrier_m}
  - k_2^- \grad p_2^-\cdot \vc n^-(x,0) &= -k_2^- \prtl_y p_2^-(x) = f^-(x) 
                    &&\text{ on }\Gamma^{-},
\end{align}
with $\Gamma^{+}$ and $\Gamma^{-}$ denoting boundary of $\Omega_2^+$ and $\Omega_2^-$, respectively, collocated with $\Omega_1$.
The communication term is:
\begin{equation}
  \label{eq:coupling_barrier}
  f(x) = f^+(x) + f^-(x),\quad f^{+/-} = \sigma^{+/-} (p^{+/-}_2(x,0) - p_1(x)).
\end{equation}
Solution to this system is discussed in Section \ref{sec:barrier_frac}. 

\section{Conductive fracture }
\label{sec:continuous_frac}
We shall derive an analytical solution to the system $(\ref{eq:Darcy_common} - \ref{eq:bc_1})$ with the coupling conditions 
$(\ref{eq:c_coupling_a} - \ref{eq:c_coupling_c})$. Symmetry of the problem in 
both $x$ and $y$ direction allows us to solve equivalent reduced problem on $\widetilde\Omega_2=(0,1)\times(0,1)$ and  
$\widetilde\Omega_1$. We impose the symmetry in $x$ direction by homegoenous Neumann condition at $x=0$. 
All equations are preserved with exception of the half flux on $\Gamma^+$. The equivalent system  reads:
\begin{align}
    \label{eq:cc_darcy_2d}
    -k_2 \Laplace p_2(x,y) &= 0         &&\text{on }\widetilde\Omega_2 \\
    \label{eq:cc_darcy_1d}
    -k_1 p_1''(x) &= f(x)               &&\text{on }\widetilde\Omega_1
\end{align}
with boundary conditions:    

\begin{equation}
    \label{eq:cc_bc_x}
    p_2(x, 1) = P_2, \quad
    k_2\prtl_y p_2(x,0) = \frac{f(x)}{2}
                        = \sigma(p_2(x,0) - p_1(x))
\end{equation}
for  $x\in (0,1)$,
\begin{equation}
    \label{eq:cc_bc_y}
    \prtl_x p_2(0,y) = \prtl_x p_2(1,y) = 0    
\end{equation}
for $y \in (0, 1)$ and finally
\begin{equation}
    \label{eq:cc_bc_1}
    p_1'(0) = 0, \quad  p_1(1) = P_1. 
\end{equation}

\begin{prop}
The solution $p_1(x)$ and $p_2(x,y)$ to the system $(\ref{eq:cc_darcy_2d} - \ref{eq:cc_bc_1})$
can be expressed in from of Fourier series:
\begin{equation}
    \label{eq:p2_series}
    p_2(x,y) = P_2 + B_0(y-1) -2B_0 \sum_{n=1}^\infty a_n \cos(\pi n x) {\rm sinh} \big(\pi n(1-y)\big)  \ \ ,
\end{equation}
%
\begin{equation}
    \label{eq:p2_series}
    p_1(x) = P_2-B_0 +u_0 \cosh(x/k) -2B_0 \sum_{n=1}^\infty  u_n \cos(\pi n x) 
\end{equation}

where we denote:
\[
    k = \sqrt{\frac{k_1}{2\sigma}}, 
\]    
and use following coefficients:

\begin{equation}
    \label{eq:an}
    a_n = \frac{(-1)^n k_2}{ k_2 \pi n \cosh(\pi n) \big(1 + (k n \pi)^2\big) 
    + \sigma (k n \pi)^2 \sinh(\pi n)} \ , 
\end{equation}

\begin{equation}
    \label{eq:u0}
    u_0 = -\frac{k_2 B_0}{\sigma k\sinh(1/k)}.
\end{equation}

\begin{equation}
    \label{eq:un}
    u_n = \frac{a_n \sinh(n \pi)}{1 + (k n \pi)^2}, 
\end{equation}




and constants:
\begin{equation}
     \label{eq:b0}
     B_0 = \frac{P_2 - P_1}{1 + 2  U + \frac{k_2}{\sigma k} \frac{\cosh(1/k)}{\sinh(1/k)}} 
\end{equation}
\begin{equation}
    \label{eq:U}
    U =  \sum_{n=1}^{\infty} (-1)^n u_n.
\end{equation}

\end{prop}

We shall apply the Fourier method \cite{??}, starting with separation of variables for the equation \eqref{eq:cc_darcy_2d}.  

\subsection{\bf Fourier method for 2D equation}
\label{sec:p2_conductive}

Considering a solution in form:
\begin{equation}
    \label{eq:sep_vars}
    p_2(x,y) = X(x)Y(y) 
\end{equation}
the equation \eqref{eq:cc_darcy_2d} gives us two equations:
\[
\frac{X''}{X} = -\frac{Y''}{Y} = L\ ,
\]
$L$ being a real constant. Applying homogeneous Neumann boundary conditions \eqref{eq:cc_bc_y}
we get possible solutions $X(x)$ in the form: 
\begin{equation}
    \label{eq:general_x}
    X_n(x) = \tilde A_n + \tilde B_n \cos (n\pi x)\qquad \text{ for }n=0,1, \dots
\end{equation}
where $\tilde A_n$, $\tilde B_n$ are arbitrary real constants and $L$ is quantizied to values:
\[
    L= - n^2 \pi^2.
\]


Next we plug $L$ in the equation for $Y(y)$ to get all solutions:
\begin{align}
Y_0(y) &= \tilde C_0 + \tilde D_0 y, \nonumber \\
\label{eq:general_y}
Y_n(y) &= \tilde C_n e^{n\pi y}+ \tilde D_n e^{-n\pi y}\qquad \text{ for } n =1, \dots
\end{align}
where $\tilde A_n, \tilde B_n$ for $n=0,1,\dots$ are arbitrary real constants.

Combining \eqref{eq:sep_vars}, \eqref{eq:general_x}, \eqref{eq:general_y} we can write the solution $p_2$ as:
\begin{equation}
    \label{eq:p2_1}
    p_2(x, y) = A_0 + B_0 y + \sum ^{\infty}_{n=1} \big(C_n + \cos (n\pi x)\big) 
            \frac{1}{2}\big(A_n e^{n\pi (y-1)} + B_n e^{-n\pi (y-1)}\big).
\end{equation}
Then the Dirichelt condition $(\ref{eq:cc_bc_x}a)$ yields:
\[
    P_2 = A_0 + B_0 + \sum ^{\infty}_{n=1} \big(C_n + \cos (n\pi x)\big) 
            \frac{A_n  + B_n}{2}.
\]
for all $x\in (0, 1)$ and thus:
\[
    A_0 + B_0 = P_2, \quad\text{ and } \quad A_n+B_n = 0.
\]
Then \eqref{eq:p2_1} can be simplified to the final form \eqref{eq:p2_series} where we yet have to determine $B_0$ and coefficients $a_n$.


\subsection{Pressure on fracture}
\label{sec:p1_conductive}

 Next step is a solution to the equation on fracture, i.e. \eqref{eq:cc_darcy_1d}. 
 Substituting for $f$ using $(\ref{eq:cc_bc_x})$ and then for $p_2$ from \eqref{eq:p2_series}
 we arrive at: 
\begin{equation}
    \label{eq:p1_equation}
    -k^2 p_1''(x) + p_1 (x) = P_2 - B_0 - 2 B_0 \sum ^{\infty}_{n=1} a_n \sinh(n\pi)\, \cos (n\pi x).
\end{equation}
with $k = \sqrt{k_1 / (2 \sigma)}$.

Solution to:
\[
    - a P''(x) + P(x) = \cos(n \pi x) 
\]    
is 
\[
    P(x) = \frac{\cos(n \pi x)}{1+an^2\pi^2}.
\]
Using linearity of the equation we obtain the general form of $p_1$ as:
\begin {equation}
    \label{eq:p1_series}        
    p_1(x) = c^+ e^{x/k} + c^- e^{-x/k} + P_2 - B_0 
    - 2 B_0 \sum^{\infty}_{n=1} u_n \cos (n\pi x).
\end {equation}
with $u_n$ given by \eqref{eq:un}. Then the boundary conditions \eqref{eq:cc_bc_1} yield:
\begin{align*}
P_1 &= p_1(1) = c^+e^{1/k} + c^- e^{-1/k} + P_2 - B_0 - 2 B_0 U, \\
0 &= p_1'(0) = \frac{1}{k}(c^+ - c^-) 
\end{align*}
with $U$ given by \eqref{eq:U}. Solving for $c^+$ and $c^-$ we get:
\[
  c^+ = c^- = \frac{P_1 - P_2 + B_0(1 + 2 U)}{2\cosh(1/k)}.
\]
Then \eqref{eq:p1_series} gives 
\begin{equation}
    \label{eq:p1_series_tmp}
    p_1(x) = P_2 -  B_0 + u_0 \cosh(x/k) - 2B_0 \sum_{n=1}^\infty  u_n \cos(\pi n x) 
\end{equation}
which is \eqref{eq:p1_series} but with 
\begin{equation}
    \label{eq:u0_def}    
    u_0 = \frac{P_1 - P_2 + B_0 + 2 B_0 U}{\cosh(1/k)}.
\end{equation}



\subsection{Resolving the coupling}
\label{sec:cont_coupling}

Last relation that we have to consider is the boundary condition $(\ref{eq:cc_bc_x}b)$. To this end we plug in the relations \eqref{eq:p2_series}, 
\eqref{eq:p1_series_tmp} for $p_2$, $p_1$  and we group the terms for remaining unknowns on the left hand side. After straight forward
manipulation we obtain:
 \[
     \frac{B_0 k_2}{\sigma} + 2 B_0 \sum_{n=1}^\infty 
         \Big[ \frac{k_2}{\sigma} n\pi  \cosh(\pi n) 
         + \sinh(\pi n)
         - \frac{\sinh(n \pi)}{1 + (k n \pi)^2} 
         \Big] a_n
         \cos(\pi n x)        
        = - u_0 \cosh(x/k) 
 \]



\begin{equation}
    \label{eq:fourier_coupling}
    \frac{{\cal A}_0}{2} + \sum^{\infty}_{n=1} {\cal A}_n \cos (n\pi x) =  - u_0 \cosh(x/k).
\end{equation}
with
\begin{equation}
    \label{eq:fourier_A0}
    {\cal A}_0 = \frac{2 k_2 B_0}{\sigma} 
\end{equation}


\begin{equation}
    \label{eq:fourier_An} 
        {\cal A}_n = 2B_0 \Bigl[ \frac{k_2}{\sigma} n\pi \cosh(n\pi) - \frac{\sinh(n \pi)}{1 + (k n \pi)^2}  
    + \sinh(n \pi) \Bigr]a_n
\end{equation}

for $n=1, \dots$. The left hand side of \eqref{eq:fourier_coupling} is the Fourier series, thus we determine
remaining unknowns $B_0$, $A_n$ by computing Fourier of the function on the right hand side and comparing coefficients.

For the zero term we have:
\[
{\cal A}_0 = -2 u_0 \int _0^1 \cosh(x/k) dx =
- 2 u_0 k\sinh 1/k.
\]
which compared to \eqref{eq:fourier_A0} gives us \eqref{eq:u0}.


For other terms, we integrate by parts to get:
\[
{\cal A}_n = -2 u_0\int _0^1 \cosh(x/k) \cos (n\pi x) dx 
= -\frac{(-1)^n 2u_0 k}{1+(k n \pi)^2}\sinh(1/k).
\]

%\[
% k_2   = \Bigl[ ({1+(k n \pi)^2}) k_2 n\pi \cosh(n\pi) + \sigma(k n \pi)^2\sinh(n \pi) \Bigr]a_n
%\]

We plug \eqref{eq:u0} into the result and compare it with \eqref{eq:fourier_An} to obtain
formula \eqref{eq:an} for coeficients $a_n$.
The only remaining unknown is $B_0$ which we determine by comparing \eqref{eq:u0_def} and \eqref{eq:u0}.
Straight forwared calculation leads to \eqref{eq:b0}.

\subsection{Evaluation of the solution}
Formulas in Theorem \ref{thm:continuous} are not suitable for practical calculations since hypergeometric functions are evaluated for large $n$.
To remadiate this issue we 


\section{Barrier fracture}
\label{sec:barrier_frac}
The case with discontinuous $p_2$ accross the fracture is solved by the similar procedure as the continuous case, 
but is slightly more technical. Considering the $y$ axis symmetry of the problem $(\ref{eq:Darcy_common} - \ref{eq:bc_1})$ and 
the coupling $(\ref{eq:bc_barrier_p} - \ref{eq:coupling_barrier})$, we can consider the same system of equations 
on reduced domains $\Omega_2^+ = (0,1)\times (0,1)$, $\Omega_2^- = (0,1)\times(-1,0)$,
$\Omega_1 = (0,1)$. To impose the symmetry we consider homogeneous Neumann boundary condition on the $y$ axis $\{(0,y)\}$.

Repeating arguments from Section \ref{sec:p2_conductive} we obtain Fourier expansion for $p_2^+$ and $p_2^-$ similar to \eqref{eq:p2_series}:
\begin{align}
    \label{eq:p2_barrier_p}
    p_2^+(x,y) &= P_2^+ + B_0^+(y-1) - u_0\sum_{n=1}^\infty a_n^+ \cos(\pi n x) {\rm sinh} \big(\pi n(1-y)\big),\\ 
    \label{eq:p2_barrier_m}
    p_2^-(x,y) &= P_2^- + B_0^-(y-1) - u_0\sum_{n=1}^\infty a_n^- \cos(\pi n x) {\rm sinh} \big(\pi n(1-y)\big).
\end{align}
with some variable $u_0$ to be specified later.

The equation for $p_1$ on $\Omega_1$ can be converted to the form similar to \eqref{eq:p1_equation}:
\begin{equation}
    \label{eq:p1_barrier_equation}
    - \ol{k}  p_1''(x) + p_1(x) = \ol{P}_2 - \ol{B}_0 - u_0\sum_{n=1}^{\infty} \ol{a}_n \cos(n \pi x) \sinh(n \pi)
\end{equation}
with help of averaged variables:

% \begin{align}
%     \label{eq:avg_sigma}
%     \ol{\sigma} &= (\sigma^+ + \sigma^-), \\
%     \label{eq:avg_k}
%     \ol{k} &= \sqrt{k_1/\ol{\sigma}},\\
%     \label{eq:avg_P2}
%     \ol{P}_2 &= \frac{\sigma^+ P_2^+ + \sigma^- P_2^-}{\ol{\sigma}}, \\
%     \label{eq:avg_B0}
%     \ol{B}_0 &= \frac{\sigma^+ B_0^+ + \sigma^- B_0^-}{\ol{\sigma}}, \\
%     \label{eq:avg_an}
%     \ol{A}_n &= \frac{\sigma^+ A_n^+ + \sigma^- A_n^-}{\ol{\sigma}}.  
% \end{align}

\begin{gather*}
    \ol{\sigma} = (\sigma^+ + \sigma^-), \qquad
    \ol{k} = \sqrt{k_1/\ol{\sigma}}, \qquad
    \ol{P}_2 = \frac{\sigma^+ P_2^+ + \sigma^- P_2^-}{\ol{\sigma}}, \\
%    
    \ol{B}_0 = \frac{\sigma^+ B_0^+ + \sigma^- B_0^-}{\ol{\sigma}}, \qquad
    \ol{a}_n = \frac{\sigma^+ a_n^+ + \sigma^- a_n^-}{\ol{\sigma}}.  
\end{gather*}

Then we can repeat derivation in Section \ref{sec:p1_conductive} to get $p_1$ expansion that closely follows
\eqref{eq:p1_series_tmp}:

\begin{equation}
    \label{eq:barrier_p1_series}
    p_1(x) = \ol{P}_2 -  \ol{B}_0 + u_0 \cosh(x/\ol{k}) - u_0 \sum_{n=1}^\infty  u_n \cos(\pi n x) 
\end{equation}
where finally identify $u_0$ as:
\begin{equation}
    \label{eq:barrier_u0_def}    
    u_0 = \frac{P_1 - \ol{P}_2 + \ol{B}_0 + u_0 U}{\cosh(1/\ol{k})},
\end{equation}
and we set:
\begin{equation}
    \label{eq:barriaer_un}
    u_n = \frac{\ol{a}_n \sinh(n \pi)}{1 + (\ol{k} n \pi)^2}, \qquad
    U =  \sum_{n=1}^{\infty} u_n.
\end{equation}

\subsection{Coupling}

Similarly as in Section \ref{sec:cont_coupling}, we plug expansion of $p_2^+$, $p_2^-$, and $p_1$ into \eqref{eq:bc_barrier_p} and \eqref{eq:bc_barrier_m}. 
We group the terms to get Fourier expansion on the left hand side and a known function of $x$ on the right hand side.
In particular on  $\Gamma^+$ we have:
% \[
%   \frac{k_2^+ B_0^+}{\sigma^+} - P_2^+ + \ol{P}_2) + B_0^+ - \ol{B}_0
%   +\sum_{n=1}^{\infty} \Big[
%         2 \frac{k_2^+ B_0^+}{\sigma^+}  a_n^+ ( n \pi)  \cosh(n \pi) 
%        + 2B_0^+  a_n^+  \sinh(\pi n)   
%        -2\ol{B}_0 u_n   
%         \Big]
%         \cos(n \pi x)  
%   = 
%    - u_0 \cosh(x/\ol{k})
% \]
% 

\begin{equation}
    \label{eq:fourier_coupling}
    \frac{{\cal A}_0^+}{2} + \sum^{\infty}_{n=1} {\cal A}_n^+ \cos (n\pi x) =  - u_0 \cosh(x/ \ol{k}).
\end{equation}
with
\begin{align}
    \label{eq:calA0_1}
    \frac{{\cal A}_0^+}{2} &= \frac{k_2^+ B_0^+}{\sigma^+} - P_2^+ + \ol{P}_2 + B_0^+ - \ol{B}_0\\
    \label{eq:calAn_1}
    {\cal A}_n^+          &=  u_0a_n^+ \Big( n \pi \frac{k_2^+ }{\sigma^+}  \cosh(n \pi) 
        +     \sinh(\pi n)  \Big) 
        - u_0u_n   
\end{align}

Reusing calculation of the Fourier coefficients of the right hand side we have: 
\begin{align}
    \label{eq:calA0_2}
    {\cal A}_0^+ &= - 2 u_0 \ol{k}\sinh( 1/\ol{k} ), \\
    \label{eq:calAn_2}
    {\cal A}_n^+ &=  -\frac{2u_0 \ol{k}}{1+(\ol{k} n \pi)^2}\sinh(1/\ol{k}).
\end{align}
Analogous relations hold for the lower side on $\Gamma^-$.

Now we combine \eqref{eq:calAn_1} and \eqref{eq:calAn_2}, we cancel $u_0$, and we substitute for $u_n$ using \eqref{eq:barriaer_un}.
Performing the same operation for the ${\cal A}_n^-$  we obtain a system to determine $a_n^+$, $a_n^-$:

\begin{equation}
    \label{eq:an_system}
    \begin{pmatrix} 
            X_n^{00} & X_n^{01} \\ 
            X_n^{10} & X_n^{11}
    \end{pmatrix}
    \begin{pmatrix} 
        a_n^+  \\ 
        a_n^-     
    \end{pmatrix}
     =  
    \begin{pmatrix} 
        y_n \\ 
        y_n
    \end{pmatrix}
\end{equation}
where
% \[
% -\frac{2u_0 \ol{k}}{1+(\ol{k} n \pi)^2}\sinh(1/\ol{k}) = A_n^+ \Big( n \pi \frac{k_2^+ }{\sigma^+}  \cosh(n \pi) 
%         +     \sinh(\pi n)  \Big) 
%         - \frac{\sigma^+ A_n^+}{\ol{\sigma}}\frac{\sinh(n \pi)}{1 + (\ol{k} n \pi)^2}
%         - \frac{\sigma^- A_n^-}{\ol{\sigma}}\frac{\sinh(n \pi)}{1 + (\ol{k} n \pi)^2}
% \]
% 
% \[
% A_n^+ \Big( n \pi \frac{k_2^+ }{\sigma^+}  \cosh(n \pi) 
%         +     \sinh(\pi n)  - \frac{\sigma^+ }{\ol{\sigma}}\frac{\sinh(n \pi)}{1 + (\ol{k} n \pi)^2}\Big)     
%         - A_n^-\frac{\sigma^- }{\ol{\sigma}}\frac{\sinh(n \pi)}{1 + (\ol{k} n \pi)^2}
%         =
%         -\frac{2u_0 \ol{k}}{1+(\ol{k} n \pi)^2}\sinh(1/\ol{k})
% \]

\begin{align}
    y_n     &= -\frac{2 \ol{k}}{1+(\ol{k} n \pi)^2}\sinh(1/\ol{k}), \\
    X_n^{00}  &= n \pi \frac{k_2^+ }{\sigma^+}  \cosh(n \pi) 
              + \sinh(\pi n)  - \frac{\sigma^+ }{\ol{\sigma}}\frac{\sinh(n \pi)}{1 + (\ol{k} n \pi)^2}, \\
    X_n^{01}  &= \frac{\sigma^- }{\ol{\sigma}}\frac{\sinh(n \pi)}{1 + (\ol{k} n \pi)^2}, \\
    X_n^{11}  &= n \pi \frac{k_2^- }{\sigma^-}  \cosh(n \pi) 
              + \sinh(\pi n)  - \frac{\sigma^- }{\ol{\sigma}}\frac{\sinh(n \pi)}{1 + (\ol{k} n \pi)^2}, \\
    X_n^{10}  &= \frac{\sigma^+ }{\ol{\sigma}}\frac{\sinh(n \pi)}{1 + (\ol{k} n \pi)^2}. 
\end{align}
The system matrix is strictly diagonally dominant providing $k_2^{+/-}$, $\sigma^{+/-}$, $k_1$ are positive.
As the right hand side is negative the coeeficients $a_n^{+/-}$ are all negative as well.


Now we express $u_0$ from \eqref{eq:barrier_u0_def}:
\[
    u_0 = \frac{P_1 - \ol{P}_2 + \ol{B}_0}{\cosh(1/\ol{k}) - U}
\]
and we plug it into  \eqref{eq:calA0_2} which we compare to \eqref{eq:calA01}. Taking a same procedure for ${\cal A}_0^-$ we
obtain a system for $B_0^+$, $B_0^-$:


\begin{align}
    \label{eq:B0_system_p}
    B_0^+\Big(\frac{k_2^+}{\sigma^+} + 1 +(T-1)\frac{\sigma^+}{\ol{\sigma}}\Big)  + B_0^-(T-1)\frac{\sigma^-}{\ol{\sigma}}
    &= P_2^+ - \ol{P}_2 +  (\ol{P}_2 - P_1) T, \\
%   
    \label{eq:B0_system_m}
    B_0^-\Big(\frac{k_2^-}{\sigma^-} + 1 +(T-1)\frac{\sigma^-}{\ol{\sigma}}\Big)  + B_0^-(T-1)\frac{\sigma^+}{\ol{\sigma}}
    &= P_2^- - \ol{P}_2 +  (\ol{P}_2 - P_1) T, 
\end{align}
denoting
\[
    T  = \frac{\ol{k}\sinh( 1/\ol{k} )}{\cosh(1/\ol{k}) - U}.
\]
The system matrix is strictly diagonally dominant as long as 
\[
    \frac{k_2^{+/-}}{\sigma^{+/-}} + 1 > 1 > T -1.
\]
Since $a_n^{+/-}$ are negative so are $u_n$ and $U$. Then
\[
    T < \ol{k}\, {\rm tanh}( 1/\ol{k} ) < 1
\]
and the system matrix is always strictly diagonally dominant.

Evaluation of the solution is performed in following steps:
\begin{enumerate}
    \item Determine number of terms $N$ in the series.
    \item Solve systems \eqref{eq:an_system} to obtain $a_n^+$, $a_n^-$ for $n=1,\dots, N$.
    \item Compute $u_n$ and the sum $U$.
    \item Solve system $(\ref{eq:B0_system_p} - \ref{eq:B0_system_m})$ for $B_0^+$, $B_0^-$.
    \item Compute $u_0$.
    \item Evaluate $p_1$ or $p_2^+$, $p_2^-$ by partial sumation of series \eqref{eq:barrier_p1_series},
          \eqref{eq:p2_series_p}, \eqref{eq:p2_series_m}, respectively.
\end{enumerate}


\section{Evaluation of series}


%%%%%%%%%%%%%%%%%%%%%%%%%%%%%%%%%%%%%%%%%

\subsubsection{Výsledné vztahy pro výpoèet tlaku}

Nyní již mùžeme vyjádøit vztahy pro výpoèet tlaku v øešené oblasti. Pro výpoèet tlaku na puklinì použijeme rovnici (\ref{r_34})

\begin{equation}
\nonumber
p_1(x) = \left(P_1 - \tilde{P}_2 + \tilde{B}_0 + \sum\limits_{n=1}^{\infty} \tilde{u}_n \right) \frac{\cosh(K(1 - x))}{\cosh K} + \tilde{P}_2 - \tilde{B}_0 - \sum\limits_{n=1}^{\infty} \tilde{u}_n \cos(n \pi x)
\end{equation}

\noindent
do které za $\tilde{u}_n$ dosadíme

\begin{equation}
\nonumber
\tilde{u}_n = \frac{\tilde{\omega} \left( 1 - e^{-2n\pi} \right) \left( \alpha^+ \sigma^+ + \alpha^- \sigma^- \right)}{2 \left( n^2 \pi^2 k_1 + \tilde{\sigma} \right)^2}
\end{equation}

\noindent
Pro výpoèet tlaku v okolním prostøedí použijeme rovnice (\ref{r_18}) a (\ref{r_19}), do kterých dosadíme (\ref{r_49}) a (\ref{r_50}). Z dùvodu pøesnìjšího výpoètu pøi implementaci vztahù do Matlabu a Pythonu nahradíme hyperbilickou funkci exponenciální. Tedy $\sinh(n\pi) = \frac{1}{2} \left( e^{n\pi} - e^{-n\pi} \right)$. Po dalších úpravách pøi kterých ze vztahu eliminujeme $e^{n\pi}$ dostaneme

\begin{equation}
\nonumber
p_2^+(x,y) = P_2^+ - B_0^+ + B_0^+y - \sum\limits_{n=1}^{\infty} \frac{\alpha^+ \tilde{\omega} }{2 \left( n^2 \pi^2 k_1 + \tilde{\sigma} \right)} \left( e^{-n\pi y} - e^{-n\pi(2-y)} \right) \cos(n\pi x )
\end{equation}

\begin{equation}
\nonumber
p_2^-(x,y) = P_2^- - B_0^- + B_0^-y - \sum\limits_{n=1}^{\infty} \frac{\alpha^- \tilde{\omega} }{2 \left( n^2 \pi^2 k_1 + \tilde{\sigma} \right)} \left( e^{-n\pi y} - e^{-n\pi(2-y)} \right) \cos(n\pi x )
\end{equation}

\noindent
Máme již vyjádøeny i vztahy pro výpoèet všech koeficientù, které je tøeba vypoèítat a následnì dosadit do koneèných vztahù pro výpoèet tlakù. Tyto koeficienty mùžeme rozdìlit na dvì skupiny. Ty, které nejsou závislé na souøadnicích ani na promìnné $n$, používané ve výpoètech souètù. Konkrétnì to jsou $B_0^+$, $B_0^-$, $\tilde{B}_0$ a  $\tilde{\omega}$. Všechny dostaneme ze soustavy rovnic (\ref{r_51}) - (\ref{r_54}) a jelikož se jedná o konstanty, staèí je vypoèítat pouze jednou. Dále pak máme koeficienty, které používáme ve výpoètech sum a jsou tedy závislé na promìnné $n$. Sem patøí $\tilde{u}_n$, $\alpha^+$ a $\alpha^-$. I ty si mùžeme pøedpoèítat a jejich hodnoty pro rùzná $n$ uložit do pole. K jejich výpoètu použijeme soustavu rovnic (\ref{r_45}) a (\ref{r_48}). K výpoètu $\tilde{u}_n$ použijeme výše uvedený vztah.

Konkrétní hodnotu tlaku v libovolném místì pukliny získáme dosazením požadované souøadnice $x$ do rovnice pro $p_1(x)$. Tlak v libovolném místì 2D oblasti získáme dosazením souøadnic $x$ a $y$ do rovnic $p_2^+(x,y)$ pro èást nad puklinou nebo do rovnice $p_2^-(x,y)$ pro èást oblasti pod puklinou.

\end{document}
